\section{Sicherheit in Freifunknetzen}
\subsection{Die kleine Übersicht}

\subsubsection{Dinge in offenen/fremden Netzwerken beachten sollte}
\shorthandoff{"}
\begin{itemize}
\item Netzwerk als "öffentliches Netzwerk" in den Windows-Firewall Einstellungen\\ markieren
\item Freigaben von Daten ins Netzwerk abschalten
\item HTTPS für Webseiten, an denen man sich anmeldet
\item SSL/TLS in E-Mail Programmen aktivieren
\item Sicherheitswarnungen im Browser beherzigen
\item (optional) Verwenden von VPN
\item Das Betriebssystem aktuell halten. \\D.h. Sicherheitsupdates regelmäßig installieren
\end{itemize}

\subsubsection{Dinge die man in offenen/fremden Netzwerken vermeiden sollte}

\begin{itemize}
\item Filesharing
\item Öffnen von Ports
\item "Firmen-/Heimnetzwerk"-Einstellungen in den Windows-Firewall Einstellungen nutzen
\end{itemize}

\subsection{WLAN Sicherheit}
\subsubsection{Was bedeutet das überhaupt?}
Bei der  "WLAN-Sicherheit"  bzw. richtiger der WLAN-Verschlüsselung handelt es 
sich ausschließlich um die Verschlüsselung, der durch die Luft übertragenen 
Daten, zwischen dem Client (Smartphone, Tablet, Laptop, etc.) und dem 
Accesspoint ( "WLAN-Router", Freifunkknoten, etc.). Hierfür stehen verschiedene 
Mechanismen zur Verfügung, die hier im Anschluss grob erklärt werden sollen.

\subsubsection{WPA/PSK und WPA2/PSK}
Bei WPA/PSK und WPA2/PSK handelt es sich um die gängigsten 
Verschlüsselungsmethoden in Deutschland. Sie sind inzwischen als Nachfolger von 
WEP die Hauptverschlüsselungsmechanismen für private WLANs. WPA beschreibt 
hierbei, wie die Nachrichten verschlüsselt werden, aber der weit wichtigere 
Teil ist das Anhängsel PSK. PSK steht für Pre Shared Key. Der Schlüssel muss 
also sowohl dem Accesspoint als auch dem Benutzer bekannt sein. Das Problem 
hierbei ist, dass \textbf{jeder} Benutzer des Netzwerks \textbf{diesen für alle gleichen} 
Schlüssel kennen muss, wodurch auch jeder, der den Schlüssel besitzt den 
gesamten WLAN Verkehr mitlesen kann.

\subsubsection{WPA/EAP und WPA2/EAP}
Hierbei handelt es sich um einen Standard, wie er in Firmen häufig verwendet 
wird. Die Verschlüsselungsmethode ist die Gleiche, wie bei privaten WLANs auch. 
Den Unterschied macht hier das EAP, welches für "Extensible Authentication 
Protocol" steht und verschiedene Mechanismen bietet die Verschlüsselung zu 
nutzen. Gängig ist hier die Anmeldung an einem WLAN, wie man es auch sonst 
gewohnt ist, über einen Usernamen und ein Passwort, aber auch ein Zugriff über 
ein Zertifikat ist möglich und hängt von der Konfiguration ab. 
 
\subsubsection{Keine (offen/unverschlüsselt)}
Nun ja, wie der Name schon sagt, findet hier einfach keine Verschlüsselung 
statt, der Zugriff ist also für jeden möglich und der Datenverkehr sichtbar.

\subsubsection{Wozu brauchen wir eigentlich eine WLAN-Verschlüsselung?}
Die WLAN-Verschlüsselung soll unsere Daten und unsere Privatssphäre schützen. 
Hierbei geht es aber weniger um die Daten, die wir direkt über das WLAN 
übertragen, denn dafür sind andere Mechanismen zuständig, als viel mehr um die 
Daten, die wir bereit stellen. 

Zum Beispiel ist es inzwischen normal geworden, dass sich in privaten Haushalten 
sogenannte NAS-Systeme befinden auf denen private Daten, wie zum Beispiel 
Urlaubsbilder und ähnliches gespeichert sind. Hier will man nicht unbedingt, dass diese 
für jeden einsehbar sind und somit soll der Zugriff darauf geschützt werden. 

Auch geschützt werden soll der Missbrauch unseres Internetanschlusses, sodass 
es eben nicht zu den Problem der Störerhaftung kommt. Hierfür verwendet Freifunk 
allerdings VPN, welches den Datenverkehr nicht direkt über den Anschluss des 
Freifunkknotenbetreibers ins Internet entlässt, sondern an die Freifunkserver 
übermittelt. So wird das Freifunknetzwerk und das Privatenetzwerk vollständig 
getrennt und das offene WLAN stellt kein Sicherheitsrisiko mehr dar. 


\subsection{Übertragungssicherheit}
\subsubsection{Was bedeutet Übertragungssicherheit?}
Bei Übertragungssicherheit, geht es darum, dass die Kommunikation zwischen dem 
Sender und dem Empfanger vor Manipulation und Auslesen geschützt wird. Hierfür 
wird eine Ende-zu-Ende-Verschlüsselung verwendet. Der meist verwendete 
Standard nennt sich SSL/TLS.

\subsubsection{Wie funktioniert SSL/TLS grundlegend?}
SSL (Secure Sockets Layer) ist die ältere Bezeichnung für das heute aktuelle 
TLS (Transport Layer Security). Hierbei geht es darum eine verschlüsselte 
Verbindung zwischen dem Client und dem Server/Dienst aufzubauen. Hierfür wird 
zunächst ausgehandelt welche Verschlüsselungen beide Seite beherrschen und 
welche für alles nachfolgende verwendet werden soll. Außerdem wird 
sichergestellt, dass die Gegenstelle auch wirklich die ist, die man erreichen 
möchte. Sobald diese Verbindung besteht werden alle Daten vollständig 
verschlüsselt zwischen den beiden Endpunkten übertragen. 

\subsubsection{HTTPS - SSL/TLS fürs Web}
Mit dieser Technologie ist vermutlich jeder schonmal in Berührung gekommen. 
Hierbei handelt es sich um die durch SSL/TLS gesicherte Variante des HTTP 
Protokolls, welches die gängige Variante ist eine Webseite aufzurufen. 

Man erkennt solche verschlüsselten Verbindungen in erster Linie an dem 
Schlosssymbol im Browser und daran, dass eine Adresse mit https:// beginnt.

Sollte zum Beispiel von einem Browser festgestellt werden, dass eine Webseite 
nicht die ist, die sie vorgibt zu sein, was durch SSL/TLS überprüft wird, so 
wird eine große, gut sichtbare Warnung angezeigt.

\subsubsection{Wozu brauchen wir SSL/TLS?}
SSL/TLS verhindert die Erfassung und Manipulation von Daten, die wir über 
Netzwerkabschnitte übertragen, die abhört werden können oder denen wir nicht 
vertrauen. Sei es ein offenes WLAN oder das Internet selbst. Dadurch, dass die 
Strecke bis zur Anwendung auf beiden Seiten verschlüsselt ist, wird so 
sichergestellt, dass keine Passwörter oder sensible Daten abhört oder 
manipuliert werden können.

